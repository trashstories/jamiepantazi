% Options for packages loaded elsewhere
\PassOptionsToPackage{unicode}{hyperref}
\PassOptionsToPackage{hyphens}{url}
\PassOptionsToPackage{dvipsnames,svgnames,x11names}{xcolor}
%
\documentclass[
  letterpaper,
  DIV=11,
  numbers=noendperiod]{scrartcl}

\usepackage{amsmath,amssymb}
\usepackage{lmodern}
\usepackage{iftex}
\ifPDFTeX
  \usepackage[T1]{fontenc}
  \usepackage[utf8]{inputenc}
  \usepackage{textcomp} % provide euro and other symbols
\else % if luatex or xetex
  \usepackage{unicode-math}
  \defaultfontfeatures{Scale=MatchLowercase}
  \defaultfontfeatures[\rmfamily]{Ligatures=TeX,Scale=1}
  \setmainfont[]{Cambria}
\fi
% Use upquote if available, for straight quotes in verbatim environments
\IfFileExists{upquote.sty}{\usepackage{upquote}}{}
\IfFileExists{microtype.sty}{% use microtype if available
  \usepackage[]{microtype}
  \UseMicrotypeSet[protrusion]{basicmath} % disable protrusion for tt fonts
}{}
\makeatletter
\@ifundefined{KOMAClassName}{% if non-KOMA class
  \IfFileExists{parskip.sty}{%
    \usepackage{parskip}
  }{% else
    \setlength{\parindent}{0pt}
    \setlength{\parskip}{6pt plus 2pt minus 1pt}}
}{% if KOMA class
  \KOMAoptions{parskip=half}}
\makeatother
\usepackage{xcolor}
\setlength{\emergencystretch}{3em} % prevent overfull lines
\setcounter{secnumdepth}{-\maxdimen} % remove section numbering
% Make \paragraph and \subparagraph free-standing
\ifx\paragraph\undefined\else
  \let\oldparagraph\paragraph
  \renewcommand{\paragraph}[1]{\oldparagraph{#1}\mbox{}}
\fi
\ifx\subparagraph\undefined\else
  \let\oldsubparagraph\subparagraph
  \renewcommand{\subparagraph}[1]{\oldsubparagraph{#1}\mbox{}}
\fi


\providecommand{\tightlist}{%
  \setlength{\itemsep}{0pt}\setlength{\parskip}{0pt}}\usepackage{longtable,booktabs,array}
\usepackage{calc} % for calculating minipage widths
% Correct order of tables after \paragraph or \subparagraph
\usepackage{etoolbox}
\makeatletter
\patchcmd\longtable{\par}{\if@noskipsec\mbox{}\fi\par}{}{}
\makeatother
% Allow footnotes in longtable head/foot
\IfFileExists{footnotehyper.sty}{\usepackage{footnotehyper}}{\usepackage{footnote}}
\makesavenoteenv{longtable}
\usepackage{graphicx}
\makeatletter
\def\maxwidth{\ifdim\Gin@nat@width>\linewidth\linewidth\else\Gin@nat@width\fi}
\def\maxheight{\ifdim\Gin@nat@height>\textheight\textheight\else\Gin@nat@height\fi}
\makeatother
% Scale images if necessary, so that they will not overflow the page
% margins by default, and it is still possible to overwrite the defaults
% using explicit options in \includegraphics[width, height, ...]{}
\setkeys{Gin}{width=\maxwidth,height=\maxheight,keepaspectratio}
% Set default figure placement to htbp
\makeatletter
\def\fps@figure{htbp}
\makeatother
\newlength{\cslhangindent}
\setlength{\cslhangindent}{1.5em}
\newlength{\csllabelwidth}
\setlength{\csllabelwidth}{3em}
\newlength{\cslentryspacingunit} % times entry-spacing
\setlength{\cslentryspacingunit}{\parskip}
\newenvironment{CSLReferences}[2] % #1 hanging-ident, #2 entry spacing
 {% don't indent paragraphs
  \setlength{\parindent}{0pt}
  % turn on hanging indent if param 1 is 1
  \ifodd #1
  \let\oldpar\par
  \def\par{\hangindent=\cslhangindent\oldpar}
  \fi
  % set entry spacing
  \setlength{\parskip}{#2\cslentryspacingunit}
 }%
 {}
\usepackage{calc}
\newcommand{\CSLBlock}[1]{#1\hfill\break}
\newcommand{\CSLLeftMargin}[1]{\parbox[t]{\csllabelwidth}{#1}}
\newcommand{\CSLRightInline}[1]{\parbox[t]{\linewidth - \csllabelwidth}{#1}\break}
\newcommand{\CSLIndent}[1]{\hspace{\cslhangindent}#1}

\usepackage{booktabs}
\usepackage{longtable}
\usepackage{array}
\usepackage{multirow}
\usepackage{wrapfig}
\usepackage{float}
\usepackage{colortbl}
\usepackage{pdflscape}
\usepackage{tabu}
\usepackage{threeparttable}
\usepackage{threeparttablex}
\usepackage[normalem]{ulem}
\usepackage{makecell}
\usepackage{xcolor}
\KOMAoption{captions}{tableheading,figureheading}
\makeatletter
\makeatother
\makeatletter
\makeatother
\makeatletter
\@ifpackageloaded{caption}{}{\usepackage{caption}}
\AtBeginDocument{%
\ifdefined\contentsname
  \renewcommand*\contentsname{Table of contents}
\else
  \newcommand\contentsname{Table of contents}
\fi
\ifdefined\listfigurename
  \renewcommand*\listfigurename{List of Figures}
\else
  \newcommand\listfigurename{List of Figures}
\fi
\ifdefined\listtablename
  \renewcommand*\listtablename{List of Tables}
\else
  \newcommand\listtablename{List of Tables}
\fi
\ifdefined\figurename
  \renewcommand*\figurename{Figure}
\else
  \newcommand\figurename{Figure}
\fi
\ifdefined\tablename
  \renewcommand*\tablename{Table}
\else
  \newcommand\tablename{Table}
\fi
}
\@ifpackageloaded{float}{}{\usepackage{float}}
\floatstyle{ruled}
\@ifundefined{c@chapter}{\newfloat{codelisting}{h}{lop}}{\newfloat{codelisting}{h}{lop}[chapter]}
\floatname{codelisting}{Listing}
\newcommand*\listoflistings{\listof{codelisting}{List of Listings}}
\makeatother
\makeatletter
\@ifpackageloaded{caption}{}{\usepackage{caption}}
\@ifpackageloaded{subcaption}{}{\usepackage{subcaption}}
\makeatother
\makeatletter
\@ifpackageloaded{tcolorbox}{}{\usepackage[many]{tcolorbox}}
\makeatother
\makeatletter
\@ifundefined{shadecolor}{\definecolor{shadecolor}{rgb}{.97, .97, .97}}
\makeatother
\makeatletter
\makeatother
\ifLuaTeX
  \usepackage{selnolig}  % disable illegal ligatures
\fi
\IfFileExists{bookmark.sty}{\usepackage{bookmark}}{\usepackage{hyperref}}
\IfFileExists{xurl.sty}{\usepackage{xurl}}{} % add URL line breaks if available
\urlstyle{same} % disable monospaced font for URLs
\hypersetup{
  pdftitle={The Price of Pretrial Detention},
  pdfauthor={Jamie Pantazi Esmond},
  colorlinks=true,
  linkcolor={blue},
  filecolor={Maroon},
  citecolor={Blue},
  urlcolor={Blue},
  pdfcreator={LaTeX via pandoc}}

\title{The Price of Pretrial Detention}
\usepackage{etoolbox}
\makeatletter
\providecommand{\subtitle}[1]{% add subtitle to \maketitle
  \apptocmd{\@title}{\par {\large #1 \par}}{}{}
}
\makeatother
\subtitle{Avoidable Impacts of Pretrial Detention on Families and
Children}
\author{Jamie Pantazi Esmond}
\date{May 29, 2023}

\begin{document}
\maketitle
\ifdefined\Shaded\renewenvironment{Shaded}{\begin{tcolorbox}[boxrule=0pt, sharp corners, interior hidden, breakable, borderline west={3pt}{0pt}{shadecolor}, frame hidden, enhanced]}{\end{tcolorbox}}\fi

\renewcommand*\contentsname{Table of contents}
{
\hypersetup{linkcolor=}
\setcounter{tocdepth}{3}
\tableofcontents
}
\listoffigures
\newpage

\hypertarget{introduction}{%
\section{Introduction}\label{introduction}}

Public safety is important, but it is often framed as a law-and-order
problem to be solved with harsher punishment for those committing
crimes. This solution, however, can have negative consequences for
families and communities leading to more crime and more harm rather than
less. Even before being convicted of a crime, people are detained in
jails, sometimes for months or years, simply because of the inability to
pay bail.

As of January 2023, Illinois became the first state to fully abolish
cash bail through the 2021 SAFE-T Act (Reichert et al., 2021). Over the
past few years, several other states, including New York, New Jersey,
and California, have reformed the practice of cash bail by eliminating
it for certain crimes (Searcy, 2022). While some opponents to these
changes worry that releasing individuals charged with a crime while they
await trial will cause an increase in crime, there is no evidence of a
correlation with increased crime rates (Preston \& Eisenberg, 2022b). In
fact, the opposite may be true; some places where cash bail has been
largely eliminated have seen lower increases in crime than the national
average or even reductions in crime when most places have seen increases
due to the COVID-19 pandemic (Covert, 2022; Preston \& Eisenberg,
2022a). Not only does cash bail not cause an increase in crime rates,
but research suggests that increased pretrial detention is linked with
higher rates of conviction and recidivism, contributing to increased
crime overall (Gupta et al., 2016; Lowenkamp, 2022). The cash bail
system and pretrial detention do not seem to meaningfully promote or
maintain public safety but may negatively influence economic and social
outcomes for families and communities.

Bail reform, including the elimination of money bail for most
misdemeanor offenses, has been shown to reduce socioeconomic inequality
in the justice system, increase compliance with future court dates, and
has saved the county and defendants millions of dollars (Widra, 2022).
These policies certainly decrease the length of time individuals are
detained, and the negative impacts of detention have been well
documented (Digard \& Swavola, 2019; W. Sawyer, 2018; Subramanian et
al., 2020). Aside from less quantifiable consequences such as job and
income loss while detained, time away from children and family, and the
impact and trauma of those relying on the individual for financial and
emotional support, there is plenty of empirical support linking pretrial
detention to higher rates of conviction (Digard \& Swavola, 2019).
Reducing the amount of time individuals spend in jail before they are
ever convicted of a crime is essential to increasing equality, reducing
factors that contribute to poverty, and reducing overall harm to
families and communities.

In 2023, almost half a million people were detained in jails pretrial in
the US (P. P. Sawyer \& Wagner, 2023). Most people detained pretrial
remain detained because of their inability to pay the bail amount.
Detaining individuals without due process is not only unconstitutional
and potentially a human rights violation, but it also negatively impacts
families and communities. According to a 2018 study, more than half of
inmates in local jails were parents to children under 18, and that
proportion was even higher for women (W. Sawyer, 2018). Guilty or not,
detaining parents harms children; the harm to children of incarcerated
parents is often difficult to quantify because it ranges from financial,
emotional, and mental health strains, to community impacts, especially
due to income and racial disparity of the communities most impacted
(Child Trends \& Justice Mapping Center, 2016).

Using a survey of adults detained due to unaffordable bail, this
research aims to illustrate the true cost of pretrial detention on
families and communities. The proposed research will explore economic
and social impacts on individuals directly impacted by pretrial
detention. Using mostly quantitative questions and options for
open-ended explanations of certain answers, a more holistic assessment
of the true benefit of eliminating cash bail can be determined as it
pertains to those directly impacted by the criminal justice process. By
investigating the impacts of pretrial detention on individuals,
policymakers can better assess the true cost of pretrial detention for
families impacted as more jurisdictions move away from the cash bail
system.

\hypertarget{literature-review}{%
\section{Literature Review}\label{literature-review}}

The Fourteenth Amendment of the US Constitution protects liberty and
equality stating, ``No State shall make or enforce any law which shall
abridge the privileges or immunities of citizens of the United States;
nor shall any State deprive any person of life, liberty, or property,
without due process of law; nor deny to any person within its
jurisdiction the equal protection of the laws'' (\emph{Fourteenth
{Amendment} to the {US Constitution}}, 1868) Cash bail systems violate
this right by imprisoning individuals who have not been convicted of a
crime simply for their inability to pay bail.

Money bail systems have long been utilized by the criminal justice
system to detain individuals arrested for a crime while they await
trial. In a study published in the Stanford Law Review, researchers
found that pretrial detention leads to individuals pleading guilty
simply to be released, receiving longer jail sentences, and it can
increase the likelihood of the individual committing future crimes
(Heaton \& Stevenson, 2016). Detaining individuals because they are poor
has a negative social and economic impact because it removes people from
their families and their jobs. By reducing negative social and economic
impacts, factors that contribute to poverty can be lessened or
eliminated. A wealth of research exists to support the theory that
poverty and inequality are positively correlated with increased crime
(Fajnzylber et al., 2002; Fleisher, 1966; Freeman, 1999; Mohammed et
al., 2018). Additionally, the issue of racial disparities in earnings is
also a well-researched subject (Akee et al., 2019; Intrator et al.,
2016; Liu et al., 2017).

Not only is much of the previous research about child and family impacts
of incarceration not up to date with ongoing trends of continuously
increasing incarceration, many either focus on long-term imprisonment or
do not distinguish between incarceration in jails pretrial or in prison
after conviction (Gabel \& Shindledecker, 1993; Miller, 2006; Parke \&
Clarke-Stewart, 2002). The recent research that does exist supports the
fact that parental incarceration has negative impacts on children's
outcomes physically, mentally, and economically (Lee et al., 2013;
Turney \& Goodsell, 2018). While some have attempted to assess the true
impact of detention on children and families, it often comes with
suggestions on mitigating the harm caused by the prison industrial
complex by providing services to affected families instead of removing
the root cause of unnecessary incarceration (Uggen \& McElrath, 2014).

In a survey conducted by George Mason University in 2016, 40.5\% of
detained parents reported that being in jail has or would change the
living situation for their children (Kimbrell \& Wilson, 2016). The
impact on children, when the parent they rely on for emotional and
financial support is separated from them, can be traumatic, significant,
and permanent (Miller, 2006). The same survey also revealed that of the
participants, 69.9\% were employed before being detained, and of those
84.3\% worried they would lose employment due to detention (Kimbrell \&
Wilson, 2016). While this study describes pretrial detainees and their
relationship with economic and family status, it lacks depth concerning
the impacts of detention on employment and childcare. Most of the
research on the impacts of incarceration relies largely on observational
data such as crime rates, unemployment, and poverty rates (Baughman,
2017; Heaton \& Stevenson, 2016; Miller, 2006).

\hypertarget{concepts-and-measurement}{%
\section{Concepts and Measurement}\label{concepts-and-measurement}}

While there are arguably many positive outcomes that could be impacted
by bail reform and the elimination of cash bail, the most important
impacts may include reduced negative social and economic impact,
reduction of factors that contribute to poverty, and reduced crime and
harm in general. Several of these outcomes have begun to garner
attention from advocates publishing data on changes after these reforms
(Heaton, 2022; Widra, 2022). Some of these impacts can be broad and
difficult to measure. However, by conducting surveys with individuals
directly impacted by pretrial detention, this research hopes to quantify
and illustrate the benefit of allowing those charged with a crime to
remain with their families, in their homes, and supported by their
communities while they await trial.

While there are many factors that contribute to poverty, the financial
and emotional stress on non-detained family members, including family
separation and job loss are measurable attributes to assess the impact
of pretrial detention. Financial and emotional stress on non-detained
family members are key contributors to poverty; many children rely on
one parent working multiple jobs to make ends meet, and emotional
support is often lacking in low-income families. Removing the only
source of income and support from children can have lasting traumatic
impacts, both financially and emotionally (Gabel \& Shindledecker, 1993;
Parke \& Clarke-Stewart, 2002; Turney \& Goodsell, 2018). Detention can
lead to worse educational outcomes, risk-taking behavior, and mental
health issues for children when separated from their parents (Lee et
al., 2013). Employment is a direct link to income; for most people,
without employment, there is no income, and lack of income directly
contributes to poverty. If individuals have jobs when they are arrested,
but lose them due to extended detention, they are in a worse position
when they are released, as are their families.

The purpose of this study is to both describe and explain the impacts of
pretrial detention. The survey component will illustrate the
characteristics of individuals directly impacted by detention and their
personal experiences. By supplementing this descriptive analysis about
who is detained with nuanced questions about the personal impact of
detention, a more holistic assessment of pretrial programs is possible.
Through regression analysis of the characteristics of people detained
pretrial and summary analysis of their personal experiences, this
research will provide insight to policymakers who are charged with
maintaining public safety and supporting impoverished communities.

When measuring social and economic impact, it is important to be
explicit about the indicators used to assess the abstract concepts that
make up social and economic well-being. To determine the rate of
household providers detained and how the people who depend on them for
financial, emotional, and physical support are impacted, a survey of
individuals detained and awaiting trial will be conducted. The unit of
analysis for the survey is individuals detained and charged with a
misdemeanor offense who were not denied bail. This is a cross-sectional
study assessing the responses of detained individuals at one point in
time during their pretrial detention. The level of measurement is
dependent on the variable being measured; to assess the scope of the
problem, dummy and interval-level variables will be used to measure
information about parental status, employment status, number of
dependents, and the length of detention. For variables about the
expected impact on children and dependents, ordinal-level responses will
be evaluated as well as open-ended answers about their family
situations.

To measure the severity of the problem, the rate of individuals detained
with minor children in their custody or family members who rely on them
for emotional and financial support will show the extent that detention
affects families. It is important to note that the level of support from
the detained parent expected from the family may differ, and the
self-reported amount of support may be inflated due to social
desirability bias. The number of individuals employed at the time of
arrest who have lost (or expect to lose) employment due to detention
will illustrate the drain on the economy overall and especially, for
those who directly rely only on income from personal employment. The
length of detention and type of arrest charge should also be considered
to isolate the impact of alleged crimes of desperation versus alleged
violent crimes.

Because the vast majority of individuals detained pre-trial are those
who cannot afford bail, it may be safe to assume that they would not
have adequate savings or other income sources for their families while
they are detained. If they did have these resources, there would be no
reason for them to remain detained. By learning how their dependents
will support themselves without them, the ripple effect of unnecessary
detention is illuminated. The lucky ones may have family members to care
for children and other dependents, but more likely, children may be
placed in the care of the state or other entity, and even if released
and exonerated, the process of regaining custody can be burdensome,
especially for low-income families of color.

The most complex and abstract concept to measure is perhaps the
emotional and physical impacts on the dependents of detained
individuals, usually children. The emotional toll of detention stretches
much further than the jail cells. Parents and caregivers provide more
than mere financial support, the psychological impacts on children with
an incarcerated parent are subtle and difficult to quantify. However, by
asking the detained parents themselves about the perceived or expected
impact of their detention on their children's emotional or mental
health, family stability, and risks to their children's outcomes, the
true total cost of pre-trial detention can begin to take shape.

While validity issues are inherent in survey research, the most
important threats to validity in this context are social desirability
bias and non-response bias. People responding to these questions are
detained may believe that their responses might influence the length of
their detention which may lead some people to exaggerate or simply lie
and provide answers that they think will improve their situation.
Although the survey administrators can remind the respondents that their
responses are anonymous and have no influence on their case, because of
the social construction of people who have been charged with a crime,
they may be inclined to present themselves as sympathetically as
possible. It is impossible to avoid this type of threat to validity
entirely in a survey of a community with such a negative social
construction, but by ensuring that the questions accurately reflect the
variables they are measuring and regard participants with the respect
they deserve as humans in a terrible situation, the threat to validity
can be reduced.

The reliability of this method is strengthened by the fact that the
conditions and surroundings of the survey process are consistent. The
survey will be administered at the jail where respondents are detained
in private rooms designated for legal counsel; the respondents will
answer the questions themselves on a tablet provided by the researcher
and in the presence of the researcher. Because the respondents are under
constant supervision in detention and generally deprived of contact with
the outside world at the time of the survey, an interaction with someone
taking an interest in their situation is likely a welcomed encounter. To
ensure consistency, the researcher will only provide standardized and
limited clarifying information about survey questions. Although, other
unavoidable reliability threats still exist, as the environment in jail
is volatile and can severely impact the disposition of respondents.

\hypertarget{methodology-and-sampling}{%
\section{Methodology and Sampling}\label{methodology-and-sampling}}

\hypertarget{subjects}{%
\subsection{Subjects}\label{subjects}}

Most bail reform policies begin by limiting or eliminating bail for
misdemeanor arrests. Without cash bail, judges retain the use of their
discretion to remand defendants deemed a threat to the community or a
flight risk. It is safe to assume that someone who is denied bail when
bail is available, would not be released without bail. The focus of this
study is people who are detained pretrial simply because they do not
have the ability to pay bail, and who would presumably be released to
await trial at home if reforms were enacted. Individuals who have been
arrested for misdemeanor offenses, have not been denied bail, and are
detained pretrial will be the subject of this research. Once the sample
of respondents is selected, the organization will contact the facility
with the list of individuals to be surveyed according to the previously
established procedure for visitation. The scope of this study will be
people detained in a single county jail over the course of four weeks.

\hypertarget{design}{%
\subsection{Design}\label{design}}

While survey research has its inherent weaknesses due to the nature of
self-reported data, to understand and attempt to measure the emotional
and social impacts of pretrial detention, it can be a useful tool to
capture a more complete assessment of the unseen costs of these
policies. By asking people directly impacted by the bail system
quantitative questions about the level of impact on their families, and
following up with the opportunity for respondents to expand on their
answers in open-ended quantitative questions about the specific
struggles their families are facing while they are detained. Though
parents can only answer these questions to the best of their knowledge
about how their families are impacted, it would be nearly impossible and
unethical to ask children themselves since being separated from a parent
or caregiver can already be a tumultuous and traumatic event for a
child. The results of these responses can be compared to existing
statistics about student performance and high-risk behavior in areas
with high arrest rates, but these statistics would be generalized to the
community and not the specific individuals and their families.

\hypertarget{measures}{%
\subsection{Measures}\label{measures}}

Measuring abstract concepts like social support, family stability,
mental health, and risk of negative outcomes offers challenges including
important validity concerns. Support or risks may look different to
different people; these concepts can be difficult to quantify. Below are
examples of the broad nature of possible questions, and these questions
will be supplemented with approximately five additional specific
questions about each concept. Using a five-point ordinal scale for each
general question and the additional specific questions, a composite of
the specific questions can be used alongside the self-reported value of
the general questions. By comparing the answers to both specific and
general questions, the validity of each is strengthened if they are
consistent.

Among the data collected, the variables hypothesized to have the most
significant relationship with family outcomes are parental status, level
of involvement, and the number of dependents; employment status or
change in employment status; and the length of detention in days. Using
regression analysis with these control variables should help to isolate
the impact of detention on the well-being of children impacted by this
system. Though the level of parental involvement cannot be asked
outright, due to certain bias in the responses, by asking a variety of
questions about the type of support, activities, and living situations
regarding their children, a composite score from one to ten can be
calculated to represent the level of involvement.

For example, to assess social support, additional questions about who is
providing care and support to dependents while they are detained. The
supplemental questions will be ordinal- or dummy-level variables, and
some may be contingent on previous answers. The level of involvement
indicated will provide a weight for some of the responses to reflect the
expected amount of impact. For example, if the parent's involvement is
high (i.e., single parent without child support) questions may be
weighted for higher representation. On the other hand, the responses
from minimally involved or absent parents (i.e., parents with children
in another state) will be weighted less. To further account for the
validity of these subjective topics, each section of questions will
conclude with an open-ended question about the topic for respondents to
provide additional information and context.

\begin{quote}
\hypertarget{sample-questions}{%
\subsubsection{Sample Questions}\label{sample-questions}}

\begin{itemize}
\tightlist
\item
  How much social \textbf{support} does your family have aside from
  yourself?

  \begin{itemize}
  \tightlist
  \item
    If you provide the primary financial support for your family, is
    there someone you know and trust who is able to ensure they have
    food and housing?
  \item
    If there is another parent involved, has their involvement
    increased?
  \end{itemize}
\item
  How would you describe your family's \textbf{stability} while you are
  detained?

  \begin{itemize}
  \tightlist
  \item
    Has your detention changed your family's housing situation?
  \item
    Has your detention changed your family's housing status?
  \item
    Is your family at a different residence while you are detained?

    \begin{itemize}
    \tightlist
    \item
      If so, are they with another immediate family member?
    \item
      An extended family member?
    \item
      A non-related friend?
    \item
      A stranger/state custody?
    \end{itemize}
  \end{itemize}
\item
  How much do you expect that your detention will negatively impact your
  child(ren)'s \textbf{emotional and mental health}?

  \begin{itemize}
  \tightlist
  \item
    How old are your child(ren)?

    \begin{itemize}
    \tightlist
    \item
      If they are young, do they understand where you are?
    \item
      If they are older, do you know how they feel about where you are?

      \begin{itemize}
      \tightlist
      \item
        If so, is it positive, negative, or indifferent?
      \end{itemize}
    \end{itemize}
  \end{itemize}
\item
  Do you think that separation from your child(ren) will increase the
  \textbf{risk of negative outcomes}?

  \begin{itemize}
  \tightlist
  \item
    If they are young, does your absence increase their exposure to
    unknown influences (different school, foster care, etc.)?
  \item
    If they are older, does your absence decrease supervision, guidance,
    or saftey for your child(ren)?
  \end{itemize}
\end{itemize}
\end{quote}

An additional issue exists concerning the nature of the respondents'
social construction, and their awareness of the negative light in which
detained people are viewed, regardless of their lack of a guilty
conviction. Detained respondents know that the general public assumes
they are guilty and would rather not think of them as people with
families and lives outside of this unfortunate situation. This reality
can lead to \emph{self-advocacy bias} in responses. Respondents may want
to appear more sympathetic by inflating or even lying outright about
their involvement or expected impact. Unfortunately, there is no way to
ethically avoid this altogether, but the strategy of asking specific
follow-up questions and open-ended questions will hopefully reduce this
possibility and help identify fraudulent responses.

\hypertarget{procedure}{%
\subsection{Procedure}\label{procedure}}

This research project consists of the following six stages: a) obtaining
approval from the Institutional Review Board (IRB), b) designing and
testing the survey, c) obtaining a sampling frame and determining the
appropriate sample size, d) collecting data via supervised
self-administered surveys, e) analyzing the data, and f) distribution of
the results.

\hypertarget{a-approval}{%
\subsubsection{a) Approval}\label{a-approval}}

Survey research has less potential for harm to participants than more
invasive methods might but considering the vulnerability of the subjects
in this study, extra care must be taken to ensure their identities
remain anonymous. While it would be useful from a data collection
perspective to request personal information about respondents in order
to follow up with them after their case is concluded to determine the
long-term implications of pretrial detention, that approach would raise
additional ethical and security concerns about the data that can be
avoided by allowing respondents to remain anonymous.

\hypertarget{b-survey-design}{%
\subsubsection{b) Survey Design}\label{b-survey-design}}

While the main topics and variables have been determined, the exact
wording and format of the survey will need to be designed and tested.
Due to the sensitive nature of some of the topics of interest, it is
important to include reviews and feedback from people who have direct
experience with the criminal justice system. This includes people who
have been detained themselves, family members of detained or formerly
detained people, and individuals who participate in the process (judges,
prosecutors, corrections officers, court clerks, etc.).

\hypertarget{c-sampling}{%
\subsubsection{c) Sampling}\label{c-sampling}}

Since the population of a county jail has continuous turnover, the
sampling frame would need to be determined each day the survey is
administered. Over the four-week period, one weekday (Monday - Thursday)
and one weekend day (Friday - Sunday) will be chosen at random to select
the survey dates. On the scheduled days, a list of individuals detained
on misdemeanor charges will be obtained from the jail, and anyone who
has been denied bail will be removed. From the final list, a sample will
be stratified by the type and number of misdemeanor charges. The types
and number of charges against individuals detained pretrial may vary
greatly; some may be as innocuous as a public intoxication charge or
even loitering, while others may involve violence or a large number of
various criminal charges. Stratification ensures representation among
varying severity and types of alleged criminal activity.

With a goal of completing approximately thirty surveys each day over
eight different visits, the complete sample should contain around 240
responses. Once the collection day sample list is determined, the list
of individuals will be sent to the facility in accordance with the
predetermined visitation policy. Non-response is an issue that may be
limited by the nature of detention; there are no prior commitments that
could take priority for potential respondents, so there is an
expectation for a high response rate. In the case that an individual
selected does not wish to participate, an additional random selection
from the strata from which they were drawn can be re-drawn to maintain
the final sample size.

\hypertarget{d-data-collection}{%
\subsubsection{d) Data Collection}\label{d-data-collection}}

The survey will be administered by one of two researchers, each with a
goal of fifteen surveys per visit. Because of security and the
bureaucratic nature of correctional facilities, delays and other
obstacles are expected. Each survey should take about 15-20 minutes to
complete. However, because these types of facilities are not known for
efficiency, to reasonably meet a goal of fifteen surveys for each
researcher, the amount of time for each response has been doubled.

Detention facilities are equipped with small private rooms for detained
individuals to meet with legal counsel and other advocates while
detained. Access to these rooms can be acquired through connections and
agreements between the organization and the facility. Each researcher
will use a tablet with a digital version of the survey. Since the tablet
must remain in the room, the survey will be answered by the respondent,
but the research will be present for the duration. In a situation where
the respondent has clarifying questions about the survey, the researcher
will be able to clarify specific things according to a predetermined
script.

\hypertarget{e-analysis}{%
\subsubsection{e) Analysis}\label{e-analysis}}

Once the data has been collected, it will need to be cleaned and
analyzed. Because the surveys were administered digitally, the responses
will be automatically uploaded. Due to the validity issues discussed
earlier, the data will be thoroughly examined to identify possible
fraudulent answers. Answers that are inconsistent or very extreme and
lack qualitative explanations in open-ended questions may be discarded.
After careful cleaning and assessment, the data will be analyzed through
multiple regression analysis including a variety of control variables to
isolate the relationship between different individual characteristics
and circumstances and the outcome variables identified.

\hypertarget{f-distribution}{%
\subsubsection{f) Distribution}\label{f-distribution}}

A final, simple report of the findings will be prepared along with
several tables and visualizations of the result for publication through
the organization and for distribution to policymakers considering
reforms to the current bail system, both in this locality and others who
are considering changes.

\hypertarget{conclusion}{%
\section{Conclusion}\label{conclusion}}

Anytime nuanced, abstract concepts are studied and quantified, there
will always be threats to validity. For as long as concepts like
well-being and stability have been studied, researchers and
decision-makers have debated how well a measure captures the essence of
the concept under scrutiny. Social desirability bias and self-advocacy
bias inherent in survey methods, especially for groups who have a
negative social construction, can further distort the results; however,
the variety of questions and supplemental open-ended responses hope to
reduce threats to validity by measuring concepts with multiple methods
for comparison. Additionally, key survey questions are based on
perceived or expected impacts since further ethical considerations exist
regarding the observation of minors who are directly impacted by
pretrial detention.

Since this survey would be distributed in only one county jail for the
purpose of this study, the results may not be generalizable to other
places. However, the study is designed so that it could easily be
reproduced in other localities with only minor modifications for broader
analysis. Since different municipalities have different policies
regarding pretrial detention and cash bail, this study would only be
appropriate for places where bail reform is needed and under
consideration.

With a design and a plan for this research in place, the next step is to
apply for IRB approval and begin conversations with the facility to
arrange for the survey distribution in person. Since both the IRB
approval and detention facility coordination processes can be tedious,
lengthy, and unpredictable, taking the initial steps should begin as
soon as possible. In completing those tasks, the proposed timeline can
be scheduled along with exact visit days and deadlines for survey
formulations and analysis. Since the surveys will be administered
digitally, the results will be immediately available. Preliminary
analysis can begin as soon as data collection begins to identify early
trends and adjust hypotheses if needed.

\begin{figure}

\caption{\label{fig-timeline}Estimated Timeline in Weeks}

{\centering \includegraphics{Bail_Reform_files/figure-pdf/fig-timeline-1.pdf}

}

\end{figure}

Since the first two tasks, obtaining IRB approval and coordinating with
the facility, are subject to timelines outside of the researchers'
control, they should be completed as soon as possible. IRB approval is
required to move forward with the study, however, the logistics between
the facility and researcher are not necessary until after the survey is
designed and tested. Figure~\ref{fig-timeline} outlines the estimated
timeline in weeks. Four weeks are allotted for IRB approval, but that
may vary depending on a number of factors. By week six, the survey
should be ready, and a schedule decided with the facility. Weeks seven
through ten is when the data will be collected (approximately thirty
surveys per day, twice a week). Before the collection of the data is
complete, preliminary analysis can begin around week nine, and a full
analysis should be complete by week twelve. The final two weeks will be
spent finalizing a written report for advocates and policymakers.

\begin{figure}

\caption{\label{fig-costs}Anticipated Expenses}

{\centering \includegraphics{Bail_Reform_files/figure-pdf/fig-costs-1.pdf}

}

\end{figure}

The cost of any research endeavor must not exceed the benefit of the
study. The paperless design of this study with digitally administered
surveys via tablets will reduce the costs and time needed to collect and
process data. The convenience of respondents having the same location at
the time of the survey also conserves resources and time.

The summary of anticipated expenses shown in Figure~\ref{fig-costs}
illustrates the breakdown of expected costs. The costs incurred for
survey design and distribution include travel to the facility, survey
software or subscription, two tablets, and wages for researchers
on-site. By far, labor costs exceed the costs of all other expenses. Two
researchers administering surveys paid \$20 per hour for eight
eight-hour days at the facility would cost \$2,560 in total. All other
expenses regarding data collection are expected to total less than \$250
(approximately \$40 for travel, \$40 for survey software, and \$150 for
tablets). An additional \$2,000 in labor costs for analysis is
equivalent to \$25 per hour for 80 hours of work from a data analyst. In
the case of unexpected incidentals, an extra \$250 has been allotted for
any unexpected expenses. The total estimated budget for this study is
\$5,040.

Once the study is complete, the final report will be published online
through the organization. A one-page summary and an infographic with key
findings will also be distributed to key policymakers and advocates for
bail reform. The infographic and other visualizations may also be shared
on social media to increase public awareness and influence.

\newpage

\hypertarget{references}{%
\section*{References}\label{references}}
\addcontentsline{toc}{section}{References}

\hypertarget{refs}{}
\begin{CSLReferences}{1}{0}
\leavevmode\vadjust pre{\hypertarget{ref-akeeRaceMattersIncome2019}{}}%
Akee, R., Jones, M. R., \& Porter, S. R. (2019). Race {Matters}: {Income
Shares}, {Income Inequality}, and {Income Mobility} for {All U}.{S}.
{Races}. \emph{Demography}, \emph{56}(3), 999--1021.
\url{https://doi.org/10.1007/s13524-019-00773-7}

\leavevmode\vadjust pre{\hypertarget{ref-baughmanCostsPretrialDetention2017}{}}%
Baughman, S. B. (2017). Costs of {Pretrial Detention}. \emph{Boston
University Law Review}, \emph{97}(1), 1--30.
\url{https://heinonline.org/HOL/P?h=hein.journals/bulr97\&i=12}

\leavevmode\vadjust pre{\hypertarget{ref-childtrendsSharedSentenceDevastating2016}{}}%
Child Trends, \& Justice Mapping Center. (2016). \emph{A {Shared
Sentence}: {The} devastating toll of parental incarceration on kids,
families and communities} {[}Policy{]}. {The Anne E. Casey Fountation}.
\url{https://assets.aecf.org/m/resourcedoc/aecf-asharedsentence-2016.pdf}

\leavevmode\vadjust pre{\hypertarget{ref-covertBailReformHelps2022}{}}%
Covert, B. (2022, July 19). \emph{Bail {Reform Helps Countless People}.
{Why Don}'t {We Hear More} of {Their Stories}?} {The Appeal}.
\url{https://theappeal.org/bail-reform-success-stories-media-coverage/}

\leavevmode\vadjust pre{\hypertarget{ref-digardJusticeDeniedHarmful2019}{}}%
Digard, L., \& Swavola, E. (2019). \emph{Justice {Denied}: {The Harmful}
and {Lasting Effects} of {Pretrial Detention}}.
\url{https://www.vera.org/downloads/publications/Justice-Denied-Evidence-Brief.pdf}

\leavevmode\vadjust pre{\hypertarget{ref-fajnzylberInequalityViolentCrime2002}{}}%
Fajnzylber, P., Lederman, D., \& Loayza, N. (2002). Inequality and
{Violent Crime}. \emph{The Journal of Law and Economics}, \emph{45}(1),
1--39. \url{https://doi.org/10.1086/338347}

\leavevmode\vadjust pre{\hypertarget{ref-fleisherEffectIncomeDelinquency1966}{}}%
Fleisher, B. M. (1966). The {Effect} of {Income} on {Delinquency}.
\emph{The American Economic Review}, \emph{56}(1/2), 118--137.
\url{http://www.jstor.org/stable/1821199}

\leavevmode\vadjust pre{\hypertarget{ref-FourteenthAmendmentUS1868}{}}%
\emph{Fourteenth {Amendment} to the {US Constitution}}. (1868). {United
States Constitution}.
\url{https://constitution.congress.gov/browse/amendment-14/}

\leavevmode\vadjust pre{\hypertarget{ref-freemanEconomicsCrime1999}{}}%
Freeman, R. B. (1999). The {Economics} of {Crime}. In \emph{Handbook of
{Labor Economics}} (Vol. 3, pp. 3529--3571). {Elsevier}.
\url{https://doi.org/10.1016/S1573-4463(99)30043-2}

\leavevmode\vadjust pre{\hypertarget{ref-gabelCharacteristicsChildrenWhose1993}{}}%
Gabel, S., \& Shindledecker, R. (1993). Characteristics of {Children
Whose Parents Have Been Incarcerated}. \emph{Psychiatric Services},
\emph{44}(7), 656--660. \url{https://doi.org/10.1176/ps.44.7.656}

\leavevmode\vadjust pre{\hypertarget{ref-guptaHeavyCostsHigh2016}{}}%
Gupta, A., Hansman, C., \& Frenchman, E. (2016). The {Heavy Costs} of
{High Bail}: {Evidence} from {Judge Randomization}. \emph{The Journal of
Legal Studies}. \url{https://doi.org/10.1086/688907}

\leavevmode\vadjust pre{\hypertarget{ref-heatonEffectsMisdemeanorBail2022}{}}%
Heaton, P. (2022). \emph{The {Effects} of {Misdemeanor Bail Reform}}.
{Quattrone Center for the Fair Administration of Justice}.
\url{https://www.law.upenn.edu/institutes/quattronecenter/reports/bailreform/\#/}

\leavevmode\vadjust pre{\hypertarget{ref-heatonDownstreamConsequencesMisdemeanor2016}{}}%
Heaton, P., \& Stevenson, M. (2016). The {Downstream Consequences} of
{Misdemeanor Pretrial Detention}. \emph{SSRN Electronic Journal}.
\url{https://doi.org/10.2139/ssrn.2809840}

\leavevmode\vadjust pre{\hypertarget{ref-intratorSegregationRaceIncome2016}{}}%
Intrator, J., Tannen, J., \& Massey, D. S. (2016). Segregation by race
and income in the {United States} 1970--2010. \emph{Social Science
Research}, \emph{60}, 45--60.
\url{https://doi.org/10.1016/j.ssresearch.2016.08.003}

\leavevmode\vadjust pre{\hypertarget{ref-kimbrellMoneyBondProcess2016}{}}%
Kimbrell, C. S., \& Wilson, D. B. (2016). \emph{Money {Bond Process
Experiences} and {Perceptions}}. {George Mason University Department of
Criminology, Law and Society}.
\url{https://www.prisonpolicy.org/scans/Money_Bond_Process_Experiences_and_Perceptions_2016.pdf}

\leavevmode\vadjust pre{\hypertarget{ref-leeImpactParentalIncarceration2013}{}}%
Lee, R. D., Fang, X., \& Luo, F. (2013). The {Impact} of {Parental
Incarceration} on the {Physical} and {Mental Health} of {Young Adults}.
\emph{Pediatrics}, \emph{131}(4), e1188--e1195.
\url{https://doi.org/10.1542/peds.2012-0627}

\leavevmode\vadjust pre{\hypertarget{ref-liuSocialCapitalRace2017}{}}%
Liu, B., Wei, Y. D., \& Simon, C. A. (2017). Social {Capital}, {Race},
and {Income Inequality} in the {United States}. \emph{Sustainability},
\emph{9}(2, 2), 248. \url{https://doi.org/10.3390/su9020248}

\leavevmode\vadjust pre{\hypertarget{ref-lowenkampHiddenCostsPretrial2022}{}}%
Lowenkamp, C. (2022). \emph{The {Hidden Costs} of {Pretrial Detention
Revisited}}. {Arnold Ventures}.
\url{https://craftmediabucket.s3.amazonaws.com/uploads/HiddenCosts.pdf}

\leavevmode\vadjust pre{\hypertarget{ref-millerImpactParentalIncarceration2006}{}}%
Miller, K. M. (2006). The {Impact} of {Parental Incarceration} on
{Children}: {An Emerging Need} for {Effective Interventions}.
\emph{Child and Adolescent Social Work Journal}, \emph{23}(4), 472--486.
\url{https://doi.org/10.1007/s10560-006-0065-6}

\leavevmode\vadjust pre{\hypertarget{ref-mohammedDoesPovertyLead2018}{}}%
Mohammed, I., Hosen, M., \& Chowdhury, M. A. F. (2018). Does poverty
lead to crime? {Evidence} from the {United States} of {America}.
\emph{International Journal of Social Economics}, \emph{45}(10),
1424--1438. \url{https://doi.org/10.1108/IJSE-04-2017-0167}

\leavevmode\vadjust pre{\hypertarget{ref-parkeEffectsParentalIncarceration2002}{}}%
Parke, R., \& Clarke-Stewart, K. A. (2002). \emph{Effects of {Parental
Incarceration} on {Young Children}} {[}Papers prepared for the "From
Prison to Home" Conference (January 30-31, 2002){]}. {U.S. Department of
Health and Human Services}.
\url{https://webarchive.urban.org/UploadedPDF/410627_ParentalIncarceration.pdf}

\leavevmode\vadjust pre{\hypertarget{ref-prestonDonBlameBail2022}{}}%
Preston, A., \& Eisenberg, R. (2022a, June 23). \emph{Don't {Blame Bail
Reform} for {Gun Violence}}. {Center for American Progress}.
\url{https://www.americanprogress.org/article/dont-blame-bail-reform-for-gun-violence/}

\leavevmode\vadjust pre{\hypertarget{ref-prestonCashBailReform2022}{}}%
Preston, A., \& Eisenberg, R. (2022b, September 19). \emph{Cash {Bail
Reform Is Not} a {Threat} to {Public Safety}}. {Center for American
Progress}.
\url{https://www.americanprogress.org/article/cash-bail-reform-is-not-a-threat-to-public-safety/}

\leavevmode\vadjust pre{\hypertarget{ref-reichert2021SAFETAct2021}{}}%
Reichert, J., Zivic, A., \& Sheley, K. (2021, July 15). \emph{The 2021
{SAFE-T Act}: {ICJIA Roles} and {Responsibilities}}. {ICJIA \textbar{}
Illinois Criminal Justice Information Authority}.
\url{https://icjia.illinois.gov/researchhub/articles/the-2021-safe-t-act-icjia-roles-and-responsibilities}

\leavevmode\vadjust pre{\hypertarget{ref-sawyerMassIncarcerationWhole2023}{}}%
Sawyer, P. P., \& Wagner, P. (2023). \emph{Mass {Incarceration}: {The
Whole Pie} 2023}. {Prison Policy Initiative}.
\url{https://www.prisonpolicy.org/reports/pie2023.html}

\leavevmode\vadjust pre{\hypertarget{ref-sawyerHowDoesUnaffordable2018}{}}%
Sawyer, W. (2018). \emph{How does unaffordable money bail affect
families?} {Prison Policy Initiative}.
\url{https://www.prisonpolicy.org/blog/2018/08/15/pretrial/}

\leavevmode\vadjust pre{\hypertarget{ref-searcyOtherStatesBail2022}{}}%
Searcy, S. (2022, October 26). \emph{Other states with bail reform}.
{WHBF - OurQuadCities.com}.
\url{https://www.ourquadcities.com/news/state-news/illinois-news/other-states-with-bail-reform/}

\leavevmode\vadjust pre{\hypertarget{ref-subramanianShadowsReviewResearch2020}{}}%
Subramanian, R., Digard, L., Washington II, M., \& Sorage, S. (2020).
\emph{In the {Shadows}: {A Review} of the {Research} on {Plea
Bargaining}}. {Vera Institute of Justice}.
\url{https://www.vera.org/downloads/publications/in-the-shadows-plea-bargaining.pdf}

\leavevmode\vadjust pre{\hypertarget{ref-turneyParentalIncarcerationChildren2018}{}}%
Turney, K., \& Goodsell, R. (2018). Parental {Incarceration} and
{Children}'s {Wellbeing}. \emph{The Future of Children}, \emph{28}(1),
147--164. \url{https://www.jstor.org/stable/26641551}

\leavevmode\vadjust pre{\hypertarget{ref-uggenParentalIncarcerationWhat2014}{}}%
Uggen, C., \& McElrath, S. (2014). Parental {Incarceration}: {What We
Know} and {Where We Need} to {Go Criminology}. \emph{Journal of Criminal
Law and Criminology}, \emph{104}(3), 597--604.
\url{https://heinonline.org/HOL/P?h=hein.journals/jclc104\&i=625}

\leavevmode\vadjust pre{\hypertarget{ref-widraWhatDoesSuccessful2022}{}}%
Widra, E. (2022). \emph{What does successful bail reform look like? {To}
start, look to {Harris County}, {Texas}.} {Prison Policy Initiative}.
\url{https://www.prisonpolicy.org/blog/2022/03/28/harris-county-pretrial-reform-results/}

\end{CSLReferences}



\end{document}
